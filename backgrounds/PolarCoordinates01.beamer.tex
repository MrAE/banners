%%%
\documentclass[xcolor=dvipsnames, aspectratio=169]{beamer}
%\usepackage{beamerthemesplit, listings, verbatim}

%\usepackage[absolute,overlay,showboxes]{textpos}
\usepackage[absolute,overlay]{textpos}

\usepackage{fontspec}

\usepackage{graphicx,wrapfig}
\usepackage{amsfonts,amsmath, amssymb, latexsym, amsthm}
\usepackage{epsfig}
\usepackage{float,enumerate}
\usepackage{listings}
\usepackage{mathtools}
\usepackage{multicol}
\usepackage{pifont}
\usepackage{tikz}
\usepackage{tikz-3dplot}
\usetikzlibrary{arrows}
\usetikzlibrary { datavisualization.formats.functions, datavisualization.polar, }
\usepackage{ulem}
\usepackage{xcolor}


\definecolor{Blueprint}{RGB}{0,61,162}
\definecolor{DarkBlue}{rgb}{0.15,0.0,0.9}
\definecolor{Red}{rgb}{0.9,0.0,0.1}
\definecolor{Colortest}{rgb}{0.1,0.55,0.1}
\definecolor{Orange}{rgb}{0.75,0.3,0.3}
\definecolor{JBlue}{HTML}{08399F}
\definecolor{Purple}{HTML}{8243DB}
\definecolor{aocbg}{HTML}{0F1022}
\definecolor{aocfg0}{HTML}{1FCA23}
\definecolor{aocfg1}{HTML}{106319}

\setbeamertemplate{navigation symbols}{} % turns nav syms off
\setbeamercolor{background canvas}{bg=aocbg}


%\title[Daily Notes]{A Beamer Template for notes}
%\subtitle[]{}
\author[JLP]{Jesse L. Patsolic}


\def\Title#1{\noindent{\large\textcolor{white}{\sf{#1}}}}
\def\MTitle#1{\noindent{\large\textcolor{white}{\tt{#1}}}}

\newcommand\Mygrid{%
\tikz[
  remember picture,
  overlay,
  color=white,
  yscale=-1,
  xstep=\TPHorizModule,ystep=\TPVertModule,
  yshift=\TPVertModule,xshift=0pt]
  \draw (current page.north west) grid (current page.south east);}

\begin{document}

%% Example of a full page figure.
\begin{frame}[plain]
%	\Mygrid
%\begin{textblock}{11}(0,0)
%    \Title{Jesse Leigh Patsolic, B.S., M.A.}
%\end{textblock}
%\makebox[\linewidth]{\includegraphics[height=0.95\paperheight]{logoBig.png}}

%%% Middle Explainer Box
%\begin{textblock}{5}(5,4)
%	\MTitle{%
%	Consider the case where $\{A, A_1\}$ are synonyms and $\{A,B\}$ are a pair of keywords.
%	}
%\end{textblock}

%\begin{textblock}{8}(0.25,15)
%    \MTitle{Polar Coordinates}%
%\end{textblock}

\begin{textblock}{12}(1.5,1)
\tikz \datavisualization [
  scientific polar axes={0 to pi, clean},
  all axes=grid,
  style sheet=vary hue,
  legend=below,
  scale=1.75,
  color=white
  ]
  [visualize as smooth line=sin,
   sin={label in legend={text=$1+\sin \alpha$}}]
  data [format=function] {
    var  angle : interval [0:pi];
    func radius = sin(\value{angle}r) + 1;
  }
  [visualize as smooth line=cos,
   cos={label in legend={text=$1+\cos\alpha$}}]
  data [format=function] {
    var  angle : interval [0:pi];
    func radius = cos(\value{angle}r) + 1;
  }
  [visualize as smooth line=cos,
   cos={label in legend={text=$1-\cos\alpha$}}]
  data [format=function] {
    var  angle : interval [0:pi];
    func radius = -cos(\value{angle}r) + 1;
  };
\end{textblock}




\end{frame}
\end{document}
%%%
%%% TIME: 
%%% WORKING STATUS:
%%% COMMENTS:
