%%%
\documentclass[xcolor=dvipsnames, aspectratio=169]{beamer}
%\usepackage{beamerthemesplit, listings, verbatim}

\usepackage[absolute,overlay,showboxes]{textpos}
%\usepackage[absolute,overlay]{textpos}

\usepackage{fontspec}

\usepackage{graphicx,wrapfig}
\usepackage{amsfonts,amsmath, amssymb, latexsym, amsthm}
\usepackage{epsfig}
\usepackage{float,enumerate}
\usepackage{listings}
\usepackage{mathtools}
\usepackage{multicol}
\usepackage{pifont}
\usepackage{tikz}
\usepackage{tikz-3dplot}
\usetikzlibrary{arrows}
\usetikzlibrary{graphs}
\usetikzlibrary{graphdrawing}
\usegdlibrary{force}
\usepackage{ulem}
\usepackage{xcolor}

\definecolor{DarkBlue}{rgb}{0.15,0.0,0.9}
\definecolor{Red}{rgb}{0.9,0.0,0.1}
\definecolor{Colortest}{rgb}{0.1,0.55,0.1}
\definecolor{Orange}{rgb}{0.75,0.3,0.3}
\definecolor{JBlue}{HTML}{08399F}
\definecolor{Purple}{HTML}{8243DB}
\definecolor{aocbg}{HTML}{0F1022}
\definecolor{aocfg0}{HTML}{1FCA23}
\definecolor{aocfg1}{HTML}{106319}
\definecolor{Peach}{HTML}{EDD1B0}


\setbeamertemplate{navigation symbols}{} % turns nav syms off
\setbeamercolor{background canvas}{bg=Peach}

%\setromanfont{BilboSwashCaps}[
%    Path=./Bilbo_Swash_Caps/,
%    Extension = .ttf,
%    UprightFont=*-Regular
%    ]

\setromanfont{MajorMonoDisplay}[
    Path=./Major_Mono_Display/,
    Extension = .ttf,
    UprightFont=*-Regular
    ]

\title[Daily Notes]{A Beamer Template for notes}
\subtitle[]{}
\author[JLP]{Jesse L. Patsolic}


\def\Title#1{\noindent{\large\textcolor{white}{\sf{#1}}}}
\def\MTitle#1{\noindent{\large\textcolor{white}{\tt{#1}}}}

\begin{document}

%% Example of a full page figure.
\begin{frame}[plain]
%\begin{textblock}{11}(0,0)
%    \Title{Jesse Leigh Patsolic, B.S., M.A.}
%\end{textblock}
%\makebox[\linewidth]{\includegraphics[height=0.95\paperheight]{logoBig.png}}

\begin{textblock}{14}(1.5,1)
	\small{\rm{
\begin{enumerate}
	\item A robot may not injure a human being or, through inaction, allow a human being to come to harm.
	\item A robot must obey the orders given it by human beings except where such orders would conflict with the First Law.
	\item A robot must protect its own existence as long as such protection does not conflict with the First or Second Law.
\end{enumerate}
}}
\end{textblock}

\begin{textblock}{14}(1.75,13.5)
Yourdon, Edward; Constantine, Larry L. (1979). Structured Design: Fundamentals of a Discipline of Computer Program and Systems Design
\end{textblock}

\end{frame}


\end{document}
%%%
%%% TIME: 
%%% WORKING STATUS:
%%% COMMENTS:
